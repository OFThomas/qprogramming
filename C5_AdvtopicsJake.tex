\section{Advanced topics}\label{Advanced topics}
1
%%%%%%%%%%%%%%%%%%%%%%%%%%%%%%%
\subsection{Quantum mechanics: The basics}

In this section we cover some supplementary background quantum theory. Although a deeper knowledge of quantum theory would serve to further your understanding of quantum computing, it should not be necessary for sections 1 to 7. You have seen in Section (?) that the state of a system governed by quantum mechanics can be represented by a binary 1 or 0, similar to that of a classical bit. However this is not the only two options. In general the state of a bit can now be in a superposition of both 1 and 0, collapsing to one or the other when measured. In this section we will cement this description into a more formal language that will help you to probe deeper into more advanced literature.

\subsubsection{Quantum States}

In quantum mechanics, we describe the state of a system simply with labels. These labels we assign to the system, such as {`0',`1'}, aim to give some intuition about the state of the system. We could equally have used {`open', `closed'} if we were trying to describe the state of a door. Formally these labels are called quantum numbers and in general are not restricted to be one of two values.\\

For example, imagine the 4 of spades was chosen from a deck of cards, a good choice of label to describe the card would be `4' or `spade'. Equally in a quantum system we would say that the card is in the state `4' or `spade' which we write formally as $\ket{4}$, or $\ket{spade}$. In this scenario the quantum number `4' could have been one of 13 values, therefore the set of states needed to fully describing the system are $\{\ket{A}, \ket{2}, \ket{3},...,\ket{K}\}$ with quantum numbers $\{A,2,3,...K\}$. If the set of quantum numbers are unique and their associated states describe all possible values a properties can take then they are said to form a Hilbert space $\mathcal{H}$ of the system. Formally we say the set of independent and orthonormal states span a Hilbert space:

\begin{equation}
\mathcal{H}:=span\{\ket{A}, \ket{2}, \ket{3},...,\ket{K}\}
\end{equation}

The Hilbert space should loosely be thought of as a vector space. It's purpose is to mathematically define all the possible states a system can be in. This is a very useful tool when we start to describe the evolution of a system because it has better be the case that our description of a system remains physically possible. For example, it would make no sense to talk about the state $\ket{A}$ evolving to the state $\ket{spade}$. In that sense, the Hilbert space helps define the boundaries of your system.\\

The notation $\ket{...}$ is called a `ket' and for every `ket' there is a `bra' written as $\bra{...}$. The names originates from the first and second halves of the word `braket', which when placed together resemble the most important operation in quantum computation: the inner product. The inner product is a simple function that does the following:\\

\textit{if state $\ket{a}$ is the same as state $\ket{b}$ return 1 else return 0}\\

In quantum physics notation this operation is performed by turning the ket $\ket{a}$ into a bra, $\bra{a}$. The `bra' $\bra{a}$ and `ket' $\ket{b}$ are then used to form the word `braket' and is equal to 1 or 0 depending on whether the state a is equal to the state b.\\

For example, returning to our deck of cards, the inner product of the states $\ket{A}$ with $\ket{5}$ is written as:

\begin{equation}
\bra{A}\ket{5} = 0
\end{equation}

Conversely, the inner product of the states $\ket{J}$ with $\ket{J}$ would be:

\begin{equation}
\bra{J}\ket{J} = 1
\end{equation}

When a set of state are unique in this way, they are said to be orthonormal. The inner product is important as it allows you to check that all the state that form your Hilbert space are orthonormal and will become very useful when we introduce superposition in the next section.

\subsubsection{Superposition}

In quantum mechanics, using the states that form the basis of our Hilbert space to describe the system is not enough. Observation of quantum systems tell us that when we prepare a state multiple times and measure it, the outcome will not always be the same. At this point its crucial to point out that the formalism does not try to explain why this is the case, but only provides a way of describing the system. I highlight this fact for the following reason. Typically when confronted with a new phenomena, we turn to the mathematical description to gain some intuition of its causes. With quantum mechanics, the formal description should not be used to find a 'logical' explanation of how superposition works but only to help fully describe the observed physics of the system.\\

To understand how we can describe this phenomena formally, lets make our deck of cards a quantum deck of cards that now obeys the laws of quantum mechanics and see how we can reflect our observations within the maths.\\

The first thing to note is that the act of measurement appears to perform an operation on the system that takes it from its superposition state to a basis state of the Hilbert space.

For example, lets say you are given a face down card that when flipped multiple times is sometimes a queen and sometimes a 4. Before performing the operation of turning it over the state of the card should be thought of as being in a superposition of $\ket{Q}$ and $\ket{4}$. Only under the operation of flipping it does it become one of the two basis states. Repeating many times would allow us to build up a picture of the probabilities of getting either one of the other. Since all we have to describe the system pre-measurement are the probabilities of getting each state after measurement, this is what we will use.\\

Formally, given a state $\ket{\psi}$ that is said to be in superposition of the states $\ket{a}$ and $\ket{b}$ where the probability of getting the state $\ket{a}$ is $\abs{\alpha}^{2}$ and the probability of getting the state $\ket{b}$ is $\abs{\beta}^{2}$ then we describe the state as:

\begin{equation}
\ket{\psi} = \alpha \ket{a} + \beta \ket{b}
\end{equation}

In general, both $\alpha$ and $\beta$ can take complex values and so to ensure that the probabilities are real we take the absolute value squared. The significance of having complex values will be discussed later on but for now its sufficient to consider them as real.\\

Returning to our example, say we get a queen one third of the time and a four two thirds of the time. We would describe the pre-measurement state as:

\begin{equation}
\ket{\psi} = \frac{1}{\sqrt{3}} \ket{Q} + \frac{2}{\sqrt{6}}  \ket{4}
\end{equation}

where

\begin{equation}
\abs{\frac{1}{\sqrt{3}}}^{2}+ \abs{\frac{2}{\sqrt{6}}}^{2} =1
\end{equation}

as expected since the probabilities of getting a queen or a four must sum to one. Consider a more general case where the state is in a superposition over all possible state in the Hilbert space. Lets call this state $\ket{\phi}$ and it is given by:

\begin{equation}
\ket{\phi} = \sum_{i=0}^{n} c_{i}\ket{\phi_{i}}
\label{equation:bigphi}
\end{equation}

% Superposition is a property of the physical object (in this case the card) which manifests itself under a measurement operation.\\


% Unlike classical representations, quantum states can be formed from superpositions of the computational basis elements of the form $|\psi\rangle = \alpha |0\rangle + \beta |1\rangle$. Taking the inner product allows us to evaluate the probability of measuring a result $\braket{\psi} = |\alpha|^{2} \braket{0}+|\beta|^{2}\braket{1}$. $\alpha$ and $\beta$ represent the weighting of the states they precede, hence P$_{0}=|\alpha|^{2}$ and P$_{1}=|\beta|^{2}$. In general, states may be formed of an arbitrarily long superposition of basis states $\ket{\phi_{n}}$, as in the example given in \ref{equation:bigphi}.

% \begin{equation}
% \ket{\Phi} = \sum_{0}^{n} c_{n}\ket{\phi_{n}}
% \label{equation:bigphi}
% \end{equation}
% where the coefficients $c_{n}$ must satisfy $\sum |c_{n}|^{2} = 1$ which can be understood as all outcome probabilities $P_{n}=|c_{n}|^{2}$ must sum to one. 
\subsubsection{Entanglement}
4 
\subsubsection{Errors \& Decoherence}
5 
%%%%%%%%%%%%%%%%%%%%%%%%%%%%%%%%
\subsection{Error correcting codes}
6
\subsection{}
