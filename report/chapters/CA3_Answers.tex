\chapter{Answers to Exercises}
\label{chpt:answers}

Not a priority right now, but to keep it in mind! :3

%%%%%%%%%%%%%%%%%%%%%%%%%%%%%%
\section{Answers for Chapter 2}

\subsection{Exercise 1:}
\subsection{Exercise 2:}
\subsection{Exercise 3:}

\newpage
%%%%%%%%%%%%%%%%%%%%%%%%%%%%%%%%
\section{Answers for Chapter 4}
%%%%%%%%%%%%%%%%%%%%%%%%%%%%%%%%

\newpage
%%%%%%%%%%%%%%%%%%%%%%%%%%%%%%%%
\section{Answers for Chapter 5}
%%%%%%%%%%%%%%%%%%%%%%%%%%%%%%%%

%%%%%%%%%%%%%%%%%%%%%%%%%%%%%%%%
%\subsection{Implementing Deutsch's Algorithm}
%%%%%%%%%%%%%%%%%%%%%%%%%%%%%%%%

%%%%%%%%%%%%%%%%%%%%%%%%%%%%%%%%%%%%%%%%%%%
\subsection{Exercise 1: Bernstein-Vazirani algorithm}
%%%%%%%%%%%%%%%%%%%%%%%%%%%%%%%%%%%%%%%%%%%

%%%%%%%%%%%%%%%%
\begin{tcolorbox}[standard jigsaw,
    opacityback=0,  % this works only in combination with the key "standard jigsaw"
    boxrule=0.5pt]
    {\bf Deutsch's Algorithm}
    \tcbline
    Given a function $f:\{0,1\}^2\rightarrow \{0,1\}$ promised to be either balanced or constant. We initialise the system in the state $\ket{00}$, we need to implement the gates $H^{\otimes 2}$ and then $U_f$ and then $H^{\otimes 2}$ and then perform a measurement in the computational basis.
\end{tcolorbox}
%%%%%%%%%%%%%%%

%%%%%%%%%%%%%%%%%
\begin{lstlisting}[language=Python,caption={Deutsch's algorithm implemented in Python only},label={lst:DApython},frame=single] 
import numpy as np
# State initialisation
state0=np.array([[1],[0],[0],[0]]) 
# Gate Implementation
H=(1/np.sqrt(2))*np.array([[1,1],[1,-1]]) 
state1=np.dot(np.kron(H,H),state0)        
U=np.array([[-1,0,0,0],
            [0,1,0,0],
            [0,0,-1,0],
            [0,0,0,1]])
state2=np.dot(U,state1)
state3=np.dot(np.kron(H,H),state2)
# Measurement: defining basis
ket0=np.array([[1],[0]])
ket1=np.array([[0],[1]])
ket00=np.kron(ket0,ket0)
ket01=np.kron(ket0,ket1)
ket10=np.kron(ket1,ket0)
ket11=np.kron(ket1,ket1)
# Measurement: projectors P00,P01,P10,P11
P00=np.dot(ket00,ket00.T)
P01=np.dot(ket01,ket01.T)
P10=np.dot(ket10,ket10.T)
P11=np.dot(ket11,ket11.T)
# Probability of obtaining: 00,01,10,11
prob00=np.trace(np.dot(P00,np.dot(state3,np.conj(state3).T)))
prob01=np.trace(np.dot(P01,np.dot(state3,np.conj(state3).T)))
prob10=np.trace(np.dot(P10,np.dot(state3,np.conj(state3).T)))
prob11=np.trace(np.dot(P11,np.dot(state3,np.conj(state3).T)))
print(prob00,prob01,prob10,prob11)
\end{lstlisting}
%%%%%%%%%%%%%%%%
%\columnbreak

Now implementing this with pyQuil. We first need to learn how to add our own gates.

%%%%%%%%%%%%%%%%%
\begin{lstlisting}[language=Python]
# Defining our own gates
Ufma=np.array([[-1,0,0,0],
               [0,1,0,0],
               [0,0,-1,0],
               [0,0,0,1]]); 
p.defgate("Uf",Ufma); 
p.inst(("Uf",0,1)); 
\end{lstlisting}
%%%%%%%%%%%%%%%%

In \autoref{lst:DAqvm}

%%%%%%%%%%%%%%%%%
\begin{lstlisting}[language=Python,caption={Deutsch's algorithm with pyQuil},label={lst:DAqvm},frame=single]
import numpy as np
from pyquil.quil import Program
from pyquil.api import QVMConnection 
from pyquil.gates import X,Z,Y,H,I 
# Invoking and renaming
qvm=QVMConnection()
p=Program() 
# Gate implementation
p.inst(H(0),H(1)) 
# Assuming the given function gives
Ufma=np.array([[-1,0,0,0],
               [0,1,0,0],
               [0,0,-1,0],
               [0,0,0,1]]);
# Adding matrix as a gate               
p.defgate("Uf",Ufma); 
# Applying new gate and Hadamards
p.inst(("Uf",0,1)); 
p.inst(H(0),H(1))
# Measurements
p.measure(0,0)
p.measure(1,1) 
# Running the program
cr=[] 
results=qvm.run(p,cr,4) 
print(results)
\end{lstlisting}
%%%%%%%%%%%%%%%%

%%%%%%%%%%%%%%%%%%%%%%%%%%%%%%%%%%%%%%%%%%%%
\subsection{Deutsch's Algorithm in Qasm (IBM Q-experience)}
%%%%%%%%%%%%%%%%%%%%%%%%%%%%%%%%%%%%%%%%%%%%
The Deutsch-Jozsa algorithm can be coded on this language as well. We begin by setting the bits in the register 
\begin{lstlisting}[language=Python,float=h]


\end{lstlisting}

\subsection{Exercise 2:}
\subsection{Exercise 3:}